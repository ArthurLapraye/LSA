\documentclass[a4paper,12pt]{article}
\usepackage[T1]{fontenc}      % un second package
\usepackage[francais]{babel}
\usepackage[utf8]{inputenc}


\author{Arthur Lapraye}

\title{Contextuals}

\begin{document}
 
 \maketitle
 

 \textit{Contextuals} est un article de Mark Aronoff publié en 1980, dont l'objet est la critique d'un article intitulé \textit{When Nouns Surface as Verbs} publié par Eve et Herbert Clark l'année précédente, lequel a pour objet l'analyse des verbes dénominaux
 de l'anglais résultant de la conversion d'un nom, et en particulier d'un sous-ensemble de ces verbes constituant des innovations :
 Clark\&Clark présentent une tripartition entre mots \textit{indexicaux} en d'autres termes, les déictiques, dont la référence est un élément du contexte d'énonciation, les \textit{dénotationnels} dont le sens est ``fixé'' dans le lexique, 
 indépendamment du contexte et les \textit{contextuals}, dont la dénotation est déterminée dans le contexte :
 "\textit{We will argue that they are neither purely denotational nor indexical, for they have a SHIFTING sense and denotation. They constitute a new category that we will call CONTEXTUALS}" (C\&C page 782 ).
	
 Aronoff critique l'introduction par C\&C de cette nouvelle catégorie dont il considère qu'elle n'est pas nécessaire pour analyser le sens des verbes en question, ainsi que le système d'interprétation présenté par C\&C
 
 1.1. AN ATTEMPT AT PERSUASION
 
 
D'après Aronoff : 
«\textit{not all unfamiliar words have the same wide range of possible meanings as these
verbs. Consider the class of novel -ly adverbs} » (page 746), faire de la familiarité l'unique critère de distinction entre \textit{contextual} et \textit{denotational} 
n'est pas bon puisque les mécanisme de dérivations par e.g suffixation donnent aussi des mots nouveau dont le sens est moins ambigu. 
Cet argument néglige le fait que précisément la dérivation par un affixe utilise un affixe connu 
et d'autant plus familier que la dérivation en question est productive.



Aronoff fait cette observation en creux juste ensuite quand il décrit la forme de la règle de conversion : 
« \textit{the meaning of the verb is limited only to an activity which has some connection with the noun.}» (p. 747) 

On voit donc que d'après son propre système, la règle de conversion est moins restreinte sémantiquement qu'une dérivation par affixation. 

 
 1.2. EVALUATIVE PHRASES
 
 
 Aronoff passe un temps considérable à montrer que le ``domaine évaluatif'' est le même pour les noms et les verbes, que les verbes n'ont pas 
 à être traités de façon particulière, mais C\&C ne disent nulle part le contraire (Aronoff a dû être trompé par le titre) 
 
   
Aronoff examine ensuite ce qu'il appelle les domaines évaluatifs : 
George is a good nurse vs George is a nurse

Selon Aronoff il est facile de construire un contexte d'interprétation pour la première phrase quand George montre une qualité d'infirmier sans en être
un, mais pas pour la deuxième phrase. 




Les contre-exemples abondent : 
``vous êtes une vraie mère pour nous'' ``he's a father to his men'', ``you be the judge'' 
à moins de supposer que seuls les contrastes ``X est Y'' vs ``X est un bon Y'' fonctionnent mais même là ça marche pas

Plus généralement, si on suppose des locuteurs infiniment coopératifs 
 
 2. DETERMINING THE EVALUATIVE DOMAIN
 
 2.1. PARAPHRASE
 2.2. ODD INTERPRETATION
 2.3. SEPARABILITé
 
 2.4. SYNTAX AND DERIVATION
 
 I will show finally that there is a major difference between these phrases
and zero-verbs: the phrases do not become fixed in their meanings as the verbs do;
they do not follow C\&C's continuum of idiomatization. This difference cannot be
accounted for by C\&C's category of contextuals. Why should only the verbs idiomatize

 3. MORPHOLOGICAL SEMANTICS AND PRAGMATICS
 
 4. THE -er-AGENTIVE
 
 5. CONCLUSION
  
 Problème dans la critique des classes, le distinguo fait par Aronoff est tout aussi arbitraire que les contextuals de C\&C 
 
 excès de subjectivité, problème d'interprétation pragmatique des éléments convertis
 
 Aronoff effleure une critique mais ne la formule pas : les locuteurs comprennent-ils vraiment toujours les contextuels ?
 
  
\end{document}
